\documentclass[12pt,a4paper,openright,twoside]{book}
\usepackage[utf8]{inputenc}
\usepackage{disi-thesis}
\usepackage{code-lstlistings}
\usepackage{notes}
\usepackage{shortcuts}
\usepackage{acronym}
\usepackage{listings-rust}
\usepackage{svg}
\usepackage{epstopdf}
\epstopdfDeclareGraphicsRule{.svg}{pdf}{.pdf}{%
    inkscape -z -D --file=#1 --export-pdf=\OutputFile
}
\AppendGraphicsExtensions{.svg}

\school{\unibo}
\programme{Corso di Laurea Magistrale in Ingegneria e Scienze Informatiche}
\title{Design and development of a Rust-based execution platform for Aggregate Computing}
\author{Micelli Leonardo}
\date{\today}
\subject{Pervasive Computing}
\supervisor{Prof. Viroli Mirko}
\cosupervisor{Dott. Aguzzi Gianluca}
\session{Quarta}
\academicyear{2022-2023}

% Definition of acronyms
\acrodef{iot}[IoT]{Internet of Things}
\acrodef{vm}[VM]{Virtual Machine}
\acrodef{ac}[AC]{Aggregate Computing}
\acrodef{fc}[FC]{Field Calculus}
\acrodef{cps}[CPS]{Cyber-Physical Systems}
\acrodef{cas}[CAS]{Collective Adaptive Systems}
\acrodef{jvm}[JVM]{Java Virtual Machine}
\acrodef{dsl}[DSL]{Domain Specific Language}
\acrodef{ast}[AST]{Abstract Syntax Tree}
\acrodef{pc}[PC]{Pervasive Computing}
\acrodef{rfid}[RFID]{Radio-Frequency Identification Technology}
\acrodef{fp}[FP]{Functional Programming}
\acrodef{scafi}[ScaFi]{Scala Fields}


\mainlinespacing{1.241} % line spacing in main matter, comment to default (1)

\begin{document}

\frontmatter\frontispiece

\begin{abstract}
    The expansion of the Internet of Things has led to the proliferation of computational resources in the physical world, which are now embedded in everyday objects and environments.
    The Aggregate Computing paradigm has emerged as a promising approach to tackle the complexity of designing and coordinating these systems, by shifting the focus from individual devices to
    programming the global behavior of whole computational collectives. However, state-of-the-art implementations of such paradigm, like ScaFi, are often based on the Java Virtual Machine, which is not suitable for all resource-constrained, ``thin'' devices.
    Therefore, the Rust Fields project aims to exploit the Rust programming language's features to provide a more lightweight and memory-efficient alternative to the JVM-based languages. In this paper, we will present the design and development
    of a Rust-based platform that will enable the distributed execution of Rust-based aggregate programs, taking a step towards the goal of bringing aggregate computing to thin devices.
\end{abstract}

\begin{dedication} % this is optional
    ‘‘The most profound technologies are those that disappear. They
    weave themselves into the fabric of everyday life until they are
    indistinguishable from it.’’
    \begin{flushright}
        (Mark Weiser, 1991)
    \end{flushright}
\end{dedication}

\begin{acknowledgements} % this is optional
    Optional. Max 1 page.
\end{acknowledgements}

%----------------------------------------------------------------------------------------
\tableofcontents
\listoffigures     % (optional) comment if empty
\lstlistoflistings % (optional) comment if empty
%----------------------------------------------------------------------------------------

\mainmatter

%----------------------------------------------------------------------------------------
% CHAPTERS
%----------------------------------------------------------------------------------------
%! Author = leona
%! Date = 09/02/24
% !TeX root = ../thesis-main.tex

\chapter{Introduction}
\label{chap:introduction}

\paragraph{Structure of the Thesis}
The structure of the thesis is designed to provide a basis for its context and objectives, and then use it as a foundation to fully describe the proposed solution.
First, a comprehensive overview of important concepts will be given in the Background section (Chapter \ref{chap:background}). Here, the reader will be introduced to the concepts
of \ac{fc}, \ac{ac} and the state of the art regarding \ac{fc} implementations, as well as the main concepts and features of the Rust programming language.
The main goal of the thesis and the requirements to achieve it will be presented in the Analysis and Requirements section (Chapter \ref{chap:requirements}).
Then, the thesis will describe the proposed solution in the Design section (Chapter \ref{chap:design}). Here, the reader will be introduced to the RuFi framework,
starting from the high-level architecture and then diving into the details of the functioning of its modules. In particular, a detailed description of notable
implementation challenges and choices will be given in the Implementation section (Chapter \ref{chap:implementation}). Finally, the Validation section (Chapter \ref{chap:validation})
will describe the validation process for the RuFi framework, which has been done on multiple axes, including unit testing, integration testing, user acceptance testing and memory profiling.

\note{At the end, describe the structure of the paper}

%! Author = leona
%! Date = 09/02/24
% !TeX root = ../thesis-main.tex

\chapter{Background}
\label{chap:background}

\section{Internet of Things and Pervasive Computing}

\subsection{The Definition of IoT}
The \ac{iot} is a broad and ever-evolving field that has its roots in the concept proposed in 1999 by Kevin Ashton where he refers to it as ``uniquely identifiable
interoperable connected objects with \ac{rfid} ''. As the years passed, the scope of IoT evolved with the rapid increase in the rate of computational power and
chip size and now it denotes an intricate, wide system of interconnected devices that become 'smart' or 'intelligent' through added sensors and computational capabilities.
Within the IoT, ``physical and virtual ‘things’ have identities and attributes and are capable of using intelligent interfaces and being integrated as an information network” (IERC 2013; Kirtsis 2011; Li et al.
2012a, b)''.

\subsection{From mobile computing to Pervasive Computing}

\section{Collective Adaptive Systems}

\section{Field Calculus}

\section{Aggregate Programming}

\section{The ScaFi Framework}

\section{The Rust Programming Language}
The Rust Programming Language is a multi-paradigm, general-purpose programming language designed originally for systems-level development. It strives to achieve both execution
speed and memory safety and efficiency while providing zero-cost abstractions and high-level features that are unusual for low-level programming languages such as C or C++.
In this section, we will go through Rust's main features and discuss how this language can be used to develop an \acs{ac} implementation that can run on thin devices.

\subsection{Rust's Basic Features}
\subsubsection{Variables and mutability}
Like the majority of today's programming languages, Rust supports storing values inside variables for referencing them in various sections of the program. \\
The developer can store a value inside a variable through a \textit{let} statement:

\begin{lstlisting}[language=Rust]
    let x = 5;
\end{lstlisting}

There is an interesting thing to notice in this piece of code: Rust is a statically typed language, but the type of the variable can be omitted thanks to the type inference mechanism. This means that the compiler can figure out the type of the variable by looking at the value assigned to it. In this case, the type of \textit{x} is \textit{i32}, which is a 32-bit signed integer. \\

Another important feature of Rust variables is that they are immutable by default. This means that once a value is assigned to a variable, it cannot be changed.
For example, the following code will not compile:

\begin{lstlisting}[language=Rust]
    let x = 5;
    x = 6; // error: cannot assign twice to immutable variable `x`
\end{lstlisting}

Instead, the developer can opt out of the mutability by using the \textit{mut} keyword:

\begin{lstlisting}[language=Rust]
    let mut x = 5;
    x = 6; // this code compiles
\end{lstlisting}

\subsubsection{Data Types}
The Rust Language supports a wide range of data types that can be both found in low-level programs and in high-level designs. These data types can be divided into two main categories: scalar types and compound types.
For scalar types, the following are supported:
\begin{itemize}
    \item \textbf{Integers}: both signed and unsigned integers of different sizes. In particular, rust supports 8, 16, 32, 64, and 128-bit signed and unsigned integers;
    \item \textbf{Floating-point numbers}: both 32 and 64-bit floating-point numbers;
    \item \textbf{Booleans}: a boolean type that can be either \textit{true} or \textit{false};
    \item \textbf{Characters}: the language's most primitive alphabetic type, represented by a single Unicode scalar value.
\end{itemize}

For compound types, the following are supported:
\begin{itemize}
    \item \textbf{Tuples}: the simplest form of product type in Rust, represented by a collection of values of possibly different types;
    \item \textbf{Arrays}: a collection of values of the same type. Unlike other languages, Rust arrays have a fixed length.
\end{itemize}

In addition to these compound types, Rust offers several other collections; for example:

\begin{itemize}
    \item \textbf{Vectors}: a collection of values of the same type. Unlike the arrays, Rust vectors have a dynamic length;
    \item \textbf{Strings}: a growable UTF-8 encoded string type;
    \item \textbf{Hash Maps}: a collection of key-value pairs, implemented as a hash table.
\end{itemize}


\subsection{The Ownership System}
Rust's Ownership System is its most unique feature and is a core part of how the language achieves memory safety without the need for a garbage collector.
The term \textit{ownership} refers to a set of rules that govern how a program's memory is managed and it is enforced by the compiler, meaning that if
a program violates them, it won't compile. This means that none of these features will cause runtime overhead for the program.

\subsubsection{Ownership Rules}
The Rust's ownership rules are the following:

\begin{enumerate}
    \item Each value in Rust has an owner.
    \item There can only be one owner at a time.
    \item When the owner goes out of scope, the value will be dropped.
\end{enumerate}

This means that a variable's validity (and presence in memory) is tied to the scope of the variable's owner: when the owner's scope is over, the compiler will automatically
call the drop function on every owned variable, freeing the memory associated with it and making it so that the variable is no longer valid.

\subsubsection{Moving and Copying}
The ownership system has implications on what happens when a variable of a certain type is copied. For example in the following code:

\begin{lstlisting}[language=Rust]
    let x = 5;
    let y = x;
\end{lstlisting}

The value of \textit{x} is copied into \textit{y}. This means that there are now two variables on the stack both with the value of 5. This is possible because x is an integer-type variable,
and integers have a fixed and known size at compile time, so they can be pushed cheaply onto the stack.

However, if we analyze the following code:

\begin{lstlisting}[language=Rust]
    let s1 = String::from("hello");
    let s2 = s1;
\end{lstlisting}

Since s1 is a String type, which does not have a known size at compile time, s1 will consist of a pointer in the stack, pointing to a heap-allocated memory that contains the actual string data, as shown in the figure \ref{fig:string-memory-rep}.

\begin{figure}[h]
    \centering
    \includegraphics[width=0.5\textwidth]{figures/string-memory-rep.png}
    \caption{Representation of the memory layout of a string in Rust}
    \label{fig:string-memory-rep}
\end{figure}

When s1 gets copied into s2, only the pointer in the stack is copied, so that the memory layout of the program will look like the one in figure \ref{fig:string-memory-rep2}.

\begin{figure}[h]
    \centering
    \includegraphics[width=0.5\textwidth]{figures/string-memory-rep-2.png}
    \caption{Representation of the memory layout of a string in Rust after the copy}
    \label{fig:string-memory-rep2}
\end{figure}

According to the ownership rules, when s1 and s2 go out of scope, one may think that the memory will be freed twice, causing a double-free error. However, in reality, after the copy, the compiler
will not consider s1 to be valid anymore, so when s1 goes out of scope, the memory will be freed only once, as shown in figure \ref{fig:string-memory-rep3}.

\begin{figure}[h]
    \centering
    \includegraphics[width=0.5\textwidth]{figures/string-memory-rep-3.png}
    \caption{Representation of the memory layout of a string in Rust after the copy and the end of the scope of s1}
    \label{fig:string-memory-rep3}
\end{figure}

In this case, it is said that the variable s1 has been \textit{moved} into s2. This means that s1 is no longer valid and cannot be used anymore.
This happens because, by default, Rust doesn't create deep copies of variables of types that don't have a known size at compile time. After all, creating a deep copy of such
a variable would cause the allocation of a new memory block on the heap, an expensive operation both in terms of execution time and memory usage. If the developer
needs to create deep copies of variables stored in the heap, they can explicitly use the \textit{clone} method, which will create a new memory block on the heap and copy the data into it.

\subsubsection{Ownership and Functions}
Similarly to what happens during the variable assignment, passing a variable to a function will cause it, depending on its type, to be moved or copied, as shown in the listing \ref{lst:func_own_1}.

\lstinputlisting[language=Rust, label={lst:func_own_1}]{listings/function_ownership_ex1.rs}

Returning values from functions will also cause ownership to be transferred, as shown in the listing \ref{lst:func_own_2}.

\lstinputlisting[language=Rust, label={lst:func_own_2}]{listings/function_ownership_ex2.rs}

\subsubsection{References and Borrowing}
Instead of taking ownership of a variable and then returning it to the caller, it is possible to pass a reference to the variable to the function, so that the function can use the variable without taking ownership of it.

\subsection{Functional Features of Rust}

\subsection{Why Rust}

\section{Towards a Rust-based AC Implementation: RuFi}
%! Author = leona
%! Date = 09/02/24
% !TeX root = ../thesis-main.tex

\chapter{Analysis and Requirements}
\label{chap:requirements}
In this chapter, we will present the goal of this thesis and describe how it will be achieved. From here, an organized structure of requirements will be presented.

\section{Thesis' Goal}
\label{sec:goal}
This thesis is within the same scope and ambit as the RustFields project, i.e. enabling the execution of aggregated programs on thin devices that would not support the JVM.
Specifically, this thesis aims to continue in this direction through the development of three objectives:

\begin{enumerate}
    \item \label{obj:1} test, validate and improve the design of the RuFi-core module;
    \item \label{obj:2} develop a new module, RuFi-distributed, that will enable the distributed execution of aggregate programs within a network of devices;
    \item \label{obj:3} develop and test an aggregate program that can be executed on a network of devices.
\end{enumerate}

\section{Requirements Breakdown Structure}
\label{sec:rbs}
From the thesis' goal and objectives, it is possible to devise a set of tasks that need to be accomplished. Each task can be further refined in sub-tasks that constitute the requirements for the main task completion.
Following this approach, a complete breakdown of the requirements emerges, as shown in this section.

\begin{enumerate}
    \item \textbf{RuFi-core}: this requirement category is related to the objective \ref{obj:1}: testing and improving the RuFi-core module.
    \begin{enumerate}
        \item \textbf{Validation}:
            \begin{enumerate}
                \item expand the "by_round" test suite;
                \item implement functions to assert the equivalence of two aggregate programs;
                \item expand the test suite with "by_equivalence" tests;
            \end{enumerate}
    \end{enumerate}
    \item \textbf{RuFi-distributed}
    \item \textbf{Aggregate Program}
\end{enumerate}

%! Author = leona
%! Date = 09/02/24
% !TeX root = ../thesis-main.tex

\chapter{Design}
\label{chap:design}
This chapter aims to give the reader a comprehensive view of this thesis' design. First, we will present the architectural design of the system, then we will shift the focus on
the detailed design, where the system will also be described in terms of behavior and interaction.

\section{Architectural Design}
\label{sec:architectural-design}
In this section, we will present and discuss RuFi's architectural design, shown in figure \ref{fig:rufi-architecture}.

\begin{figure}[ht]
    \centering
    \includegraphics[width=0.8\textwidth]{figures/diagrams/img/rufi-architecture.png}
    \caption{RuFi's architectural design}
    \label{fig:rufi-architecture}
\end{figure}

In the figure, we can see the following components:

\begin{itemize}
    \item \textbf{rf-core}: this component defines key abstractions, such as the fundamental aggregate operators, builtins and a virtual machine;
    \item \textbf{rf-distributed}: this component defines concepts related to the distributed execution of aggregate programs.
          In particular, it defines core abstractions related to networking and message passing, as well as an implementation of the computational model discussed in \ref{par:comp-model};
    \item \textbf{rf-distributed-impl}: this component exposes a standard implementation for the concepts defined in \textbf{rf-distributed}.
          The choice of separating the abstraction definitions and the implementations will be discussed in section \ref{subsec:rufi-distributed};
    \item \textbf{rf-gradient}: this component is a library exposing the gradient algorithm as an aggregate program.
    \item \textbf{rufi}: this component, which serves as the main user interface with the framework, re-exports all the other components under a coherent namespace.
\end{itemize}

\subsection{RuFi Core}

\subsection{RuFi Distributed}
\label{subsec:rufi-distributed}

\section{Detailed Design}
\label{sec:detailed-design}

\subsection{Behavior}

\subsection{Interaction}
%! Author = leona
%! Date = 09/02/24
% !TeX root = ../thesis-main.tex

\chapter{Implementation}
\label{chap:Implementation}

\section{Crate Structure}
%TODO parla della struttura del crate e di come importare tutto in un nuovo progetto

\section{RuFi Core}

\section{RuFi Distributed}
%! Author = leona
%! Date = 09/02/24
% !TeX root = ../thesis-main.tex

\chapter{Validation}
\label{chap:validation}

The validation for this thesis' work has been done on multiple axes, which will be described in this chapter.

\section{Unit Testing}
The first layer of testing is the ``Unit Testing''. In computer science, this term refers to the act of analyzing and scrutinizing the smallest units of software possible.
This thesis adheres to Rust's unit testing practices: each public module that is part of the library has a corresponding and private testing module, annotated with the
conditional compilation macro ``\#[cfg(test)]'', denoting this is a module that is only compiled when the test suite is run.\\
Inside this module, it is possible to write test functions by annotating them with the ``\#[test]'' attribute. These functions can then be run with the ``cargo test'' command.\\
As such, each module in the RuFi library crates has a corresponding testing module containing unit test functions for them. For example, the listing \ref{lst:unit_test} shows the unit tests for the
vm\_status module.

\lstinputlisting[language=Rust, label={lst:unit_test}]{listings/unit_test.rs}

\section{Integration Testing}
Unit testing is a crucial practice in the development of software artifacts, but testing each component in isolation is not sufficient to analyze every aspect of the software produced.
Another important practice is the act of testing some or many components together, to ensure that their behavior when interacting is the one expected, which is called ``Integration Testing''.
Again, this thesis adheres to Rust's integration testing practices: each library crate has a corresponding ``tests'' directory, where integration tests are written. These tests are run with the ``cargo test'' command, just like their unit counterpart\\
Inside the tests directory, it is possible to create various source files TODO

\section{Memory Profiling}

\section{Quality Control}


%----------------------------------------------------------------------------------------
% BIBLIOGRAPHY
%----------------------------------------------------------------------------------------

\backmatter

\nocite{*} % comment this to only show the referenced entries from the .bib file

\bibliographystyle{alpha}
\bibliography{bibliography}

\end{document}