%! Author = leona
%! Date = 09/02/24
% !TeX root = ../thesis-main.tex

\chapter{Introduction}
\label{chap:introduction}

\paragraph{Structure of the Thesis}
The structure of the thesis is designed to provide a basis for its context and objectives, and then use it as a foundation to fully describe the proposed solution.
First, a comprehensive overview of important concepts will be given in the Background section (Chapter \ref{chap:background}). Here, the reader will be introduced to the concepts
of \ac{fc}, \ac{ac} and the state of the art regarding \ac{fc} implementations, as well as the main concepts and features of the Rust programming language.
The main goal of the thesis and the requirements to achieve it will be presented in the Analysis and Requirements section (Chapter \ref{chap:requirements}).
Then, the thesis will describe the proposed solution in the Design section (Chapter \ref{chap:design}). Here, the reader will be introduced to the RuFi framework,
starting from the high-level architecture and then diving into the details of the functioning of its modules. In particular, a detailed description of notable
implementation challenges and choices will be given in the Implementation section (Chapter \ref{chap:implementation}). Finally, the Validation section (Chapter \ref{chap:validation})
will describe the validation process for the RuFi framework, which has been done on multiple axes, including unit testing, integration testing, user acceptance testing and memory profiling.

\note{At the end, describe the structure of the paper}
