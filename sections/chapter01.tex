%! Author = leona
%! Date = 09/02/24
% !TeX root = ../thesis-main.tex

\chapter{Introduction}
\label{chap:introduction}
The current trends from \ac{iot} are ushering in a new era for the interaction between humans and computing devices. The steady growth in the number of objects that are embedded with computational power
and connection capabilities presents numerous opportunities and challenges. This environment has favored the birth of a new vision of the future of computing: \ac{pc} \cite{satyanarayanan2001pervasive}.
According to this vision, computational resources, which are now deeply intertwined with the physical world to the point of being almost invisible, operate and coordinate in an increasingly
dynamic environment, where decentralized and peer-to-peer interactions are expected to be increasingly important. In this world, the \ac{ac} paradigm offers an encouraging shift of focus on the design
of the systems of the future, where the emphasis will now be on the global behavior of collections of devices, rather than on the individual devices themselves \cite{viroli2019aggregate}.
The development of this new field has already led to the creation of very important and well-researched computational models, languages and frameworks. However, the current state of the art in \ac{ac}
frameworks is for the most part based on the Java Virtual Machine ...

\paragraph{Structure of the Thesis}
The structure of the thesis is designed to provide a basis for its context and objectives, and then use it as a foundation to fully describe the proposed solution.
First, a comprehensive overview of important concepts will be given in the Background section (Chapter \ref{chap:background}). Here, the reader will be introduced to the concepts
of \ac{fc}, \ac{ac} and the state of the art regarding \ac{fc} implementations, as well as the main concepts and features of the Rust programming language.
The main goal of the thesis and the requirements to achieve it will be presented in the Analysis and Requirements section (Chapter \ref{chap:requirements}).
Then, the thesis will describe the proposed solution in the Design section (Chapter \ref{chap:design}). Here, the reader will be introduced to the RuFi framework,
starting from the high-level architecture and then diving into the details and functioning of its modules. In particular, a thorough description of notable
implementation challenges and choices made will be given in the Implementation section (Chapter \ref{chap:implementation}). Finally, the Validation section (Chapter \ref{chap:validation})
will describe the validation process for the RuFi framework, which has been done on multiple axes, including unit testing, integration testing, user acceptance testing and memory profiling.
