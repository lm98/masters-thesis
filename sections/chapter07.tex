\chapter{Conclusion}
\label{chap:conclusions}
In this thesis, we have presented the design and development of a distributed execution platform for the RustFields framework, which has the ultimate goal of bringing Aggregate Computing to resource-constrained devices.
Starting by establishing a solid base regarding the context, paradigms and state-of-the-art for Aggregate Computing and a solid foundation of the Rust programming language concepts and idioms,
we were able to identify a set of requirements and goals for this thesis project, which were then used as a guide during the design and development phases.
Our analysis of the current state of the RustFields project has highlighted the need for validation and improvement of the current core library, as well as the need for a new module that would enable
the distributed execution of Rust-based aggregate programs. With this consideration in mind, we have proposed the design of the new RustFields framework, RuFi, highlighting first the architectural design of the project and its main
components, and then delving into the detailed design of such components. These designs were then used as a guideline for the implementation phase, where we highlighted some important tactical choices that were made.
We also started collecting experimental data on memory usage of the current RustFields framework, which will be useful for future research and improvements.

\section{Current Limitations}
As of now, although the main goal of the thesis of providing a distributed execution platform for the RustFields framework has been achieved, the higher-level objective of bringing \ac{ac} on all thin devices is still not fully accomplished.
Experiments on running the current RuFi framework in very resource-constrained devices like the Esp32 have shown that the current implementation is not yet suitable for such devices, as the memory usage is still too high, highlighting the need
for further research and improvement on the memory footprint of such a framework.

\section{Future Work}
The previous analysis of the limitations of the current RuFi implementation has already suggested an important objective for future work.
Nevertheless, there are also other interesting directions to consider, such as:

\begin{itemize}
    \item implementation of a proprietary Mqtt broker: currently, RuFi relies on a public Mqtt broker for the communication between devices.
          However, having a proprietary Mqtt broker would allow for more control over the communication and the possibility of implementing more advanced features;
    \item implementation of a more advanced Domain Specific Language: although the topic was briefly addressed during the development of the thesis, it played a marginal role.
          However, it would be interesting in the future to be able to go into more detail about the design and development of a macro-based DSL that would allow for a better developer experience when writing and reading aggregate programs written with RuFi;
\end{itemize}